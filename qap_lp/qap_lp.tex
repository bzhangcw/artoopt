%%
% Copyright (c) 2017 - 2020, Pascal Wagler;
% Copyright (c) 2014 - 2020, John MacFarlane
%
% All rights reserved.
%
% Redistribution and use in source and binary forms, with or without
% modification, are permitted provided that the following conditions
% are met:
%
% - Redistributions of source code must retain the above copyright
% notice, this list of conditions and the following disclaimer.
%
% - Redistributions in binary form must reproduce the above copyright
% notice, this list of conditions and the following disclaimer in the
% documentation and/or other materials provided with the distribution.
%
% - Neither the name of John MacFarlane nor the names of other
% contributors may be used to endorse or promote products derived
% from this software without specific prior written permission.
%
% THIS SOFTWARE IS PROVIDED BY THE COPYRIGHT HOLDERS AND CONTRIBUTORS
% "AS IS" AND ANY EXPRESS OR IMPLIED WARRANTIES, INCLUDING, BUT NOT
% LIMITED TO, THE IMPLIED WARRANTIES OF MERCHANTABILITY AND FITNESS
% FOR A PARTICULAR PURPOSE ARE DISCLAIMED. IN NO EVENT SHALL THE
% COPYRIGHT OWNER OR CONTRIBUTORS BE LIABLE FOR ANY DIRECT, INDIRECT,
% INCIDENTAL, SPECIAL, EXEMPLARY, OR CONSEQUENTIAL DAMAGES (INCLUDING,
% BUT NOT LIMITED TO, PROCUREMENT OF SUBSTITUTE GOODS OR SERVICES;
% LOSS OF USE, DATA, OR PROFITS; OR BUSINESS INTERRUPTION) HOWEVER
% CAUSED AND ON ANY THEORY OF LIABILITY, WHETHER IN CONTRACT, STRICT
% LIABILITY, OR TORT (INCLUDING NEGLIGENCE OR OTHERWISE) ARISING IN
% ANY WAY OUT OF THE USE OF THIS SOFTWARE, EVEN IF ADVISED OF THE
% POSSIBILITY OF SUCH DAMAGE.
%%

%%
% This is the Eisvogel pandoc LaTeX template.
%
% For usage information and examples visit the official GitHub page:
% https://github.com/Wandmalfarbe/pandoc-latex-template
%%


% @modified: Chuwen <chuwzhang@gmail.com>
% Options for packages loaded elsewhere
\PassOptionsToPackage{unicode}{hyperref}
\PassOptionsToPackage{hyphens}{url}
\PassOptionsToPackage{dvipsnames,svgnames*,x11names*,table}{xcolor}
%
\documentclass[
  10pt,
  a4paper,
,tablecaptionabove
]{scrartcl}
\usepackage{lmodern}
\usepackage{setspace}
\setstretch{1.2}
\usepackage{amssymb,amsmath}
\usepackage{ifxetex,ifluatex}
\ifnum 0\ifxetex 1\fi\ifluatex 1\fi=0 % if pdftex
  \usepackage[T1]{fontenc}
  \usepackage[utf8]{inputenc}
  \usepackage{textcomp} % provide euro and other symbols
\else % if luatex or xetex
  \usepackage{unicode-math}
  \defaultfontfeatures{Scale=MatchLowercase}
  \defaultfontfeatures[\rmfamily]{Ligatures=TeX,Scale=1}
\fi
% Use upquote if available, for straight quotes in verbatim environments
\IfFileExists{upquote.sty}{\usepackage{upquote}}{}
\IfFileExists{microtype.sty}{% use microtype if available
  \usepackage[]{microtype}
  \UseMicrotypeSet[protrusion]{basicmath} % disable protrusion for tt fonts
}{}
\makeatletter
\@ifundefined{KOMAClassName}{% if non-KOMA class
  \IfFileExists{parskip.sty}{%
    \usepackage{parskip}
  }{% else
    \setlength{\parindent}{0pt}
    \setlength{\parskip}{6pt plus 2pt minus 1pt}}
}{% if KOMA class
  \KOMAoptions{parskip=half}}
\makeatother
\usepackage{xcolor}
\definecolor{default-linkcolor}{HTML}{A50000}
\definecolor{default-filecolor}{HTML}{A50000}
\definecolor{default-citecolor}{HTML}{4077C0}
\definecolor{default-urlcolor}{HTML}{4077C0}
\IfFileExists{xurl.sty}{\usepackage{xurl}}{} % add URL line breaks if available
\IfFileExists{bookmark.sty}{\usepackage{bookmark}}{\usepackage{hyperref}}
\hypersetup{
  pdftitle={QAP},
  pdfauthor={Chuwen Zhang},
  hidelinks,
  breaklinks=true,
  pdfcreator={LaTeX via pandoc with the Eisvogel template}}
\urlstyle{same} % disable monospaced font for URLs
\usepackage[margin=2.5cm,includehead=true,includefoot=true,centering,]{geometry}
% add backlinks to footnote references, cf. https://tex.stackexchange.com/questions/302266/make-footnote-clickable-both-ways
\usepackage{footnotebackref}
\setlength{\emergencystretch}{3em}  % prevent overfull lines
\providecommand{\tightlist}{%
  \setlength{\itemsep}{0pt}\setlength{\parskip}{0pt}}
\setcounter{secnumdepth}{3}

% Make use of float-package and set default placement for figures to H.
% The option H means 'PUT IT HERE' (as  opposed to the standard h option which means 'You may put it here if you like').
\usepackage{float}
\floatplacement{figure}{H}


\usepackage[UTF8, heading=true]{ctex}
\definecolor{tufeijilk}{RGB}{68,87,151}
\hypersetup{colorlinks=true,linkcolor=tufeijilk,urlcolor=cyan}
\newlength{\cslhangindent}
\setlength{\cslhangindent}{1.5em}
\newenvironment{cslreferences}%
  {}%
  {\par}

\title{QAP}
\author{Chuwen Zhang}
\date{}



%%
%% added
%%

%
% language specification
%
% If no language is specified, use English as the default main document language.
%

\ifnum 0\ifxetex 1\fi\ifluatex 1\fi=0 % if pdftex
  \usepackage[shorthands=off,main=english]{babel}
\else
  % @update, do not use sfdefault,
  % @chuwen, 20200711
  %   % % Workaround for bug in Polyglossia that breaks `\familydefault` when `\setmainlanguage` is used.
  % % See https://github.com/Wandmalfarbe/pandoc-latex-template/issues/8
  % % See https://github.com/reutenauer/polyglossia/issues/186
  % % See https://github.com/reutenauer/polyglossia/issues/127
  % \renewcommand*\familydefault{\sfdefault}
  %   % load polyglossia as late as possible as it *could* call bidi if RTL lang (e.g. Hebrew or Arabic)
  \usepackage{polyglossia}
  \setmainlanguage[]{english}
\fi



%
% for the background color of the title page
%

%
% break urls
%
\PassOptionsToPackage{hyphens}{url}

%
% When using babel or polyglossia with biblatex, loading csquotes is recommended
% to ensure that quoted texts are typeset according to the rules of your main language.
%
\usepackage{csquotes}

%
% captions
%
\definecolor{caption-color}{HTML}{777777}
\usepackage[font={stretch=1.2}, textfont={color=caption-color}, position=top, skip=4mm, labelfont=bf, singlelinecheck=false, justification=raggedright]{caption}
\setcapindent{0em}

%
% blockquote
%
\definecolor{blockquote-border}{RGB}{221,221,221}
\definecolor{blockquote-text}{RGB}{119,119,119}
\usepackage{mdframed}
\newmdenv[rightline=false,bottomline=false,topline=false,linewidth=3pt,linecolor=blockquote-border,skipabove=\parskip]{customblockquote}
\renewenvironment{quote}{\begin{customblockquote}\list{}{\rightmargin=0em\leftmargin=0em}%
\item\relax\color{blockquote-text}\ignorespaces}{\unskip\unskip\endlist\end{customblockquote}}

%
% heading color
%
\definecolor{heading-color}{RGB}{40,40,40}
\addtokomafont{section}{\color{heading-color}}
% When using the classes report, scrreprt, book,
% scrbook or memoir, uncomment the following line.
%\addtokomafont{chapter}{\color{heading-color}}

%
% variables for title and author
%
\usepackage{titling}
\title{QAP}
\author{Chuwen Zhang}

%
% tables
%

%
% remove paragraph indention
%
\setlength{\parindent}{0pt}
\setlength{\parskip}{6pt plus 2pt minus 1pt}
\setlength{\emergencystretch}{3em}  % prevent overfull lines

%
%
% Listings
%
%


%
% header and footer
%
\usepackage{fancyhdr}

\fancypagestyle{eisvogel-header-footer}{
  \fancyhead{}
  \fancyfoot{}
  \lhead[]{QAP}
  \chead[]{}
  \rhead[QAP]{}
  \lfoot[\thepage]{Chuwen Zhang}
  \cfoot[]{}
  \rfoot[Chuwen Zhang]{\thepage}
  \renewcommand{\headrulewidth}{0.4pt}
  \renewcommand{\footrulewidth}{0.4pt}
}
\pagestyle{eisvogel-header-footer}

%%
%% end added
%%

\begin{document}

%%
%% begin titlepage
%%

%%
%% end titlepage
%%



\hypertarget{qap-the-problem}{%
\section{QAP, the problem}\label{qap-the-problem}}

QAP, and alternative descriptions, see
\protect\hyperlink{ref-jiang_l_p-norm_2016}{1}

\[\begin{aligned}
&\min_X f(X) = \textrm{tr}(A^\top XB X^\top)  \\
& = \textrm{tr}(X^\top A^\top XB) & x = \textrm{vec}(X)\\
& = \left <\textrm{vec}(X),  \textrm{vec}(A^\top X B )  \right > \\
& = \left <\textrm{vec}(X), B^\top \otimes A^\top \cdot \textrm{vec}(X)  \right > \\ 
& = x^\top (B^\top \otimes A^\top) x\\ 
\mathbf{s.t.} & \\ 
&X \in \Pi_{n}
\end{aligned}\]

is the optimization problem on permutation matrices:

\[ \Pi_{n}=\left\{X \in \mathbb R ^{n \times n} \mid X e =X^{\top} e = e , X_{i j} \in\{0,1\}\right\}\]

The convex hull of permutation matrices, the Birkhoff polytope, is
defined:

\[D _{n}=\left\{X \in \mathbb R ^{n \times n} \mid X e =X^{\top} e = e , X \geq 0 \right\}\]

for the constraints, also equivalently: \[\begin{aligned}
& \textrm{tr}(XX^\top) = \left <x, x \right >_F= n, X \in D_{n}
\end{aligned}\]

\hypertarget{differentiation}{%
\subsection{Differentiation}\label{differentiation}}

\[\begin{aligned}
&  \nabla f = A^\top XB + AXB^\top \\
& \nabla \textrm{tr}(XX^\top) = 2X
\end{aligned}\]

\hypertarget{mathscr-l_p-regularization}{%
\section{\texorpdfstring{\(\mathscr L_p\)
regularization}{\textbackslash mathscr L\_p regularization}}\label{mathscr-l_p-regularization}}

various form of regularized problem:

\begin{itemize}
\item
  \(\mathscr L_0\): \(f(X) + \sigma ||X||_0\) is exact to the original
  problem for efficiently large \(\sigma\)
  \protect\hyperlink{ref-jiang_l_p-norm_2016}{1}, but the problem itself
  is still NP-hard.
\item
  \(\mathscr L_p\): also suggested by
  \protect\hyperlink{ref-jiang_l_p-norm_2016}{1}, good in the sense:

  \begin{itemize}
  \tightlist
  \item
    strongly concave and the global optimizer must be at vertices
  \item
    \textbf{local optimizer is a permutation matrix} if
    \(\sigma, \epsilon\) satisfies some condition. Also, there is a
    lower bound for nonzero entries of the KKT points
  \end{itemize}
\end{itemize}

\[\min _{X \in D _{n}} F_{\sigma, p, \epsilon}(X):=f(X)+\sigma\|X+\epsilon 1 \|_{p}^{p}\]

\begin{itemize}
\tightlist
\item
  \(\mathscr L_2\), and is based on the fact that
  \(\Pi_n = D_n \bigcap \{X:\textrm{tr}(XX^\top) = n\}\),
  \protect\hyperlink{ref-xia_efficient_2010}{2}
\end{itemize}

\[\min_Xf(X)+\mu_{0} \cdot \textrm{tr} \left(X X^{\top}\right)\]

\hypertarget{mathscr-l_2}{%
\subsection{\texorpdfstring{\(\mathscr L_2\)}{\textbackslash mathscr L\_2}}\label{mathscr-l_2}}

\hypertarget{naive}{%
\subsubsection{naive}\label{naive}}

\[\begin{aligned}
&\textrm{tr}(A^\top XB X^\top) + \mu_0 \cdot \textrm{tr}(X X^{\top}) \\
= & x^\top (B^\top \otimes A^\top + \mu\cdot  \mathbf e_{n\times n}) x\\ 
\end{aligned} \] this implies a LD-like method. (but not exactly)

\hypertarget{mathscr-l_1-exact-penalty}{%
\subsection{\texorpdfstring{\(\mathscr L_1\) exact
penalty}{\textbackslash mathscr L\_1 exact penalty}}\label{mathscr-l_1-exact-penalty}}

Motivated by the formulation using trace:

\[\begin{aligned}
& \min_X  f \\
\mathbf{s.t.} &\\
&   \textrm{tr}(XX^\top ) -  n = 0 \\
& X \in D_n
\end{aligned}\]

using \(\mathscr L_1\) and by the factor that
\(\forall X \in D_n ,\; \textrm{tr}(XX^\top)\le n\), we have:

\[\begin{aligned}
F_{\mu} & =  f  + \mu\cdot | \textrm{tr}(XX^\top ) -  n| \\
 &= f  + \mu\cdot n - \mu\cdot \textrm{tr}(XX^\top )
\end{aligned}\]

For sufficiently large penalty parameter \(\mu\), the problem solves the
original problem.

The penalty method is very likely to become a concave function (even if
the original one is convex), and thus it cannot be directly solved by
conic solver.

\hypertarget{projected-gradient}{%
\subsubsection{Projected gradient}\label{projected-gradient}}

Suppose we do projection on the penalized problem \(F_\mu\) \#\#\#\#
derivatives

\[\begin{aligned}
& \nabla_X F_\mu  = A^\top XB + AXB^\top - 2\mu X \\
& \nabla_\mu F_\mu  = n - \textrm{tr}(XX^\top) \\
& \nabla_\Lambda F_\mu  = - X
\end{aligned}\]

\hypertarget{projected-derivative}{%
\paragraph{projected derivative}\label{projected-derivative}}

\(PD\), a quadratic program

\[\begin{aligned}
&\min_D ||\nabla F_\mu + D ||_F^2  \\
\mathbf{s.t.} & \\
&D e = D^\top e = 0 \\ 
&D_{ij} = 0 \quad \textsf{if: } X_{ij} = 0\\
\end{aligned}\]

facts:

the space of \(D\), (\(e\) is the vector of 1)

\[D \in \{D\in\mathbb{R}^{n\times n} : \; D e = D^\top e = 0;\; D_{ij} = 0,\;\forall  (i,j) \in M \}\]

how to formulate for \(F\) such that \(\left <F, D \right>_F = 0\) ?

\(\mathbf I\) is the identity matrix for active constraints of the
\(X \ge 0\) where \(\mathbf I_{ij} = 1\) if \(X_{ij} = 0\)

\begin{itemize}
\tightlist
\item
  \(\left <D + \nabla F_\mu, D \right > = 0\)
\end{itemize}

dual problem for \(PD\)

\begin{itemize}
\tightlist
\item
  \(\alpha,\beta,\Lambda\) are Lagrange multipliers, \(\mathbf I\) is
  the identity matrix for active constraints of the \(X \ge 0\) where
  \(\mathbf I_{ij} = 1\) if \(X_{ij} = 0\)
\end{itemize}

\[\begin{aligned}
& L_d = 1/2\cdot ||\nabla F_\mu + D ||_F^2 - \alpha^\top De - \beta^\top D^\top e -\Lambda \bullet D \bullet \mathbf I \\
\mathsf{KKT:} &\\
& \nabla F+D - ae^\top - e\beta^\top -\Lambda \bullet \mathbf{I} = 0\\
& \nabla Fe - ae^\top e - e\beta^\top e -\Lambda \bullet \mathbf{I} e = 0\\
& \nabla F^\top e  - \beta e^\top e - e\alpha^\top e - (\Lambda \bullet \mathbf{I})^\top e = 0
\end{aligned}\]

\hypertarget{section}{%
\subsubsection{}\label{section}}

\hypertarget{reference}{%
\section*{Reference}\label{reference}}
\addcontentsline{toc}{section}{Reference}

\hypertarget{refs}{}
\begin{cslreferences}
\leavevmode\hypertarget{ref-jiang_l_p-norm_2016}{}%
{[}1{]} B. Jiang, Y.-F. Liu, and Z. Wen, ``L\_p-norm regularization
algorithms for optimization over permutation matrices,'' \emph{SIAM
Journal on Optimization}, vol. 26, no. 4, pp. 2284--2313, 2016.

\leavevmode\hypertarget{ref-xia_efficient_2010}{}%
{[}2{]} Y. Xia, ``An efficient continuation method for quadratic
assignment problems,'' \emph{Computers \& Operations Research}, vol. 37,
no. 6, pp. 1027--1032, 2010.
\end{cslreferences}

\end{document}
